\documentclass{article}\usepackage[]{graphicx}\usepackage[]{color}
%% maxwidth is the original width if it is less than linewidth
%% otherwise use linewidth (to make sure the graphics do not exceed the margin)
\makeatletter
\def\maxwidth{ %
  \ifdim\Gin@nat@width>\linewidth
    \linewidth
  \else
    \Gin@nat@width
  \fi
}
\makeatother

\definecolor{fgcolor}{rgb}{0.345, 0.345, 0.345}
\newcommand{\hlnum}[1]{\textcolor[rgb]{0.686,0.059,0.569}{#1}}%
\newcommand{\hlstr}[1]{\textcolor[rgb]{0.192,0.494,0.8}{#1}}%
\newcommand{\hlcom}[1]{\textcolor[rgb]{0.678,0.584,0.686}{\textit{#1}}}%
\newcommand{\hlopt}[1]{\textcolor[rgb]{0,0,0}{#1}}%
\newcommand{\hlstd}[1]{\textcolor[rgb]{0.345,0.345,0.345}{#1}}%
\newcommand{\hlkwa}[1]{\textcolor[rgb]{0.161,0.373,0.58}{\textbf{#1}}}%
\newcommand{\hlkwb}[1]{\textcolor[rgb]{0.69,0.353,0.396}{#1}}%
\newcommand{\hlkwc}[1]{\textcolor[rgb]{0.333,0.667,0.333}{#1}}%
\newcommand{\hlkwd}[1]{\textcolor[rgb]{0.737,0.353,0.396}{\textbf{#1}}}%

\usepackage{framed}
\makeatletter
\newenvironment{kframe}{%
 \def\at@end@of@kframe{}%
 \ifinner\ifhmode%
  \def\at@end@of@kframe{\end{minipage}}%
  \begin{minipage}{\columnwidth}%
 \fi\fi%
 \def\FrameCommand##1{\hskip\@totalleftmargin \hskip-\fboxsep
 \colorbox{shadecolor}{##1}\hskip-\fboxsep
     % There is no \\@totalrightmargin, so:
     \hskip-\linewidth \hskip-\@totalleftmargin \hskip\columnwidth}%
 \MakeFramed {\advance\hsize-\width
   \@totalleftmargin\z@ \linewidth\hsize
   \@setminipage}}%
 {\par\unskip\endMakeFramed%
 \at@end@of@kframe}
\makeatother

\definecolor{shadecolor}{rgb}{.97, .97, .97}
\definecolor{messagecolor}{rgb}{0, 0, 0}
\definecolor{warningcolor}{rgb}{1, 0, 1}
\definecolor{errorcolor}{rgb}{1, 0, 0}
\newenvironment{knitrout}{}{} % an empty environment to be redefined in TeX

\usepackage{alltt}
\IfFileExists{upquote.sty}{\usepackage{upquote}}{}
\begin{document}



\section{Introducing my Country Coder}

Use country coder - uses fuzzy matching to chose assign a standardized name to the country.
\begin{knitrout}
\definecolor{shadecolor}{rgb}{0.969, 0.969, 0.969}\color{fgcolor}\begin{kframe}
\begin{alltt}
\hlkwd{options}\hlstd{(}\hlkwc{width}\hlstd{=}\hlnum{80}\hlstd{)}
\hlkwd{source}\hlstd{(}\hlstr{"countrycode2.R"}\hlstd{)}

\hlcom{#Example of how country coder works: }
\hlstd{StandardNames}\hlkwb{=}\hlkwd{FavoriteCountryName}\hlstd{(}\hlkwd{c}\hlstd{(}\hlstr{"Ireland"}\hlstd{,}
                                    \hlstr{"United Kingdom"}\hlstd{,}
                                    \hlstr{"United Kingdom of Great Britain and Northern Ireland"}\hlstd{,}
                                    \hlstr{"United States of America"}\hlstd{,}
                                    \hlstr{"United States"}\hlstd{,}
                                    \hlstr{"iraq"}\hlstd{,}
                                    \hlstr{"Iran"}\hlstd{,}
                                    \hlstr{"Iran"}\hlstd{,}
                                    \hlstr{"Ceylon"}\hlstd{))}
\end{alltt}
\begin{verbatim}
## [1] "The unique input and output are:"
##          Unique.Inputs.Truncated            Out
## 1                        Ireland        Ireland
## 2                 United Kingdom United Kingdom
## 3 United Kingdom of Great Britai United Kingdom
## 4       United States of America  United States
## 5                  United States  United States
## 6                           iraq           Iraq
## 7                           Iran           Iran
## 8                         Ceylon      Sri Lanka
\end{verbatim}
\begin{alltt}
\hlstd{StandardNames}
\end{alltt}
\begin{verbatim}
## [1] "Ireland"        "United Kingdom" "United Kingdom" "United States" 
## [5] "United States"  "Iraq"           "Iran"           "Iran"          
## [9] "Sri Lanka"
\end{verbatim}
\end{kframe}
\end{knitrout}



\end{document}
